%% start of file `template_en.tex'.
%% Copyright 2006-2008 Xavier Danaux (xdanaux@gmail.com).
%
% This work may be distributed and/or modified under the
% conditions of the LaTeX Project Public License version 1.3c,
% available at http://www.latex-project.org/lppl/.

\documentclass[11pt,a4paper]{moderncv}


\moderncvtheme[green]{classic}

% character encoding
\usepackage{ucs}
\usepackage[utf8x]{inputenc}
\usepackage[english,romanian]{babel}


\usepackage[scale=0.8]{geometry}
\setlength{\hintscolumnwidth}{3cm}
\AtBeginDocument{
	\recomputelengths 
	\hypersetup{
		colorlinks,
		citecolor=violet,
		linkcolor=red,
		urlcolor=blue
		}
	}

% personal data
\firstname{Alexandru \mbox{George}}
\familyname{\mbox{Burghelea}}

\title{Curriculum Vitae}               

\address{Strada Cometei numarul 5}{810354 Brăila, Romania}
\mobile{+40 743 061 489}
\email{\mbox{aburghelea@rosedu.org}}
\photo[60pt]{cv_photo_2012.jpg}
\quote{Un program care aproape merge e ca un avion care aproape zboara! Nistor Moț}

%\nopagenumbers{}

%----------------------------------------------------------------------------------
%            content
%----------------------------------------------------------------------------------
\begin{document}
\maketitle

\section{Experienta}
\cventry{Oct. 2012 \newline --Ian. 2013}{Asistent Asociat}{Proiectare Orientata pe Obiecte, Universitatea Politehnica din Bucuresti, Facultatea de Automatica si Calculatoare}{\newline Am fost responsabil de a explica studentilor conceptele OOP, sa ii ajut in rezolvarea exercitiilor de laborator precum si corectarea si evaluarea temelor}{}{Tehnologii folosite: Java SE, Swing}

\cventry{Oct. 2011 \newline--present}{Programator junior Java}{TeamNet Solutions S.A., Bucuresti}{\newline 
Am lucrat la o aplicatie unde principala functionalitate este un motor de cautare peste un warehouse. Inainte am lucrat la o aplicatie proiectata pentru
procesarea si manipularea datelor. De asemenea am am lucrat la un tool pentru importarea si procesarea automata a datelor, conform unei logici de business
din aplicatii deja existente}{}{Tehnologii folosite: Groovy on Grails, jQuery, Selenium, Spring, Solr, Nutch}
\cventry{Jun. 2011\newline --Sept.2011}{Stagiar}{TeamNet International S.A., Bucuresti }
	{\newline 
	Am lucrat la noi functionalitati si am reparat probleme deja existente intr-o aplicatie care proceseaza cererile de finantare din partea Uniunii Europene}{}{Tehnologii folosite: Java,Hibernate, Seam, Birt}

\section{Education}
\cventry{2009--present}{Engineer}{Politehnica University of Bucharest, Automatic Control and Computer Science}{Bucharest}{major Computer Science and Engineering}{}{}
\cventry{2005--2009}{Bacalaureat}{National High-school "Nicolae Balcescu"}{Braila}{Specialization Mathematics and Informatics. GPA 9.66 out of 10}{}{}
 
\section{Computer skills}
\cvcomputer{Advanced}{Java, C}{}{}
\cvcomputer{Intermediate}{\mbox{Groovy on Grails, C++, Selenium, Hibernate}}{}{}
\cvcomputer{Basic}{\mbox{Spring, Seam, Wicket, Birt Report Builder, Shell Scripting}}{}{}
\cvcomputer{Operating Systems}{\mbox{Linux(Ubuntu, Fedora), Windows}}{}{}
\cvcomputer{Miscellaneous}{Office, Autocad, Verilog, PSpice}{}{}

\section{Contests}
\cvline{National Olympiad in Informatics}{participant in 2008,2009}
\cvline{Severin Bumbaru}{1\textsuperscript{st} place in 2009}
\cvline{Urmasii lui Moisil}{participant in 2008}
\cvline{Infoeducatie}{participant in 2005}

\section{Languages}
\cvlanguage{Romanian}{Advanced}{First Language}
\cvlanguage{English}{Advanced}{}

\section{Interest and activities}
\cventry{Twivi}{Social Coding}{TNI}{}{}{Along with 3 friends we entered a 10 hour hackathon where we ranked 3\textsuperscript{rd}. Twivi was designed to consume the Twitter feed and based on the keywords the user selected from the feed would generate videos after extracting images from Wikipedia.}
\cventry{Volunteering}{ROSEdu}{Course Organizing}{}{}{I am volunteering at \href{http://www.rosedu.org}{Romanian Open Source Education} organization, where I have helped organize a \href{http://webdev.rosedu.org}{Web Development Course} and an \href{http://workshop.rosedu.org/wiki/sesiuni/algoritmica}{Algorithm Workshop}}
\cventry{FDC}{Freeware Development Course}{ROSEdu}{}{}{I participated as a student at FDC where I learned how to use version control systems and other technologies, while developing, with two other students, a plugin for Pidgin, similar to Yahoo Doodle!.}
\cventry{Scientific Circle}{Physics Department}{Politehnica University of Bucharest}{}{}{I participated at the Scientific Circle in 2009, where my team ranked 2\textsuperscript{nd}, with a paper on the Galilean Satellites. We also generated a video of their orbits by using a C++ program for solving Kepler's Equations.}
\cventry{Facebook Match}{Object Oriented Programming Course}{Politehnica University of Bucharest}{}{}{I had to implement in Java, using Facebook API , an application that would create different statistics based on the activity of a user.}
\cventry{Driving AI}{Algorithm Design Course}{Politehnica University of Bucharest}{}{}{Along with 3 friends, we implemented in C++ an engine that could drive a car on a circuit. Based on the information received from the server, the application calculated the acceleration/break and direction where the car should go.}
\cventry{Chat client}{Communication Protocols}{Politehnica University of Bucharest}{}{}{I implemented a Linux CLI LAN chat client, based on a client-server model. The application had was able to send messages, transfer files and also had group messages}

\section{Hobbies}
\cvline{}{Airplanes, Swimming, Driving, Hiking, Music}

\section{References}
\cvline{}{Available upon request}

\end{document}
%% end of file `template_en.tex'.
